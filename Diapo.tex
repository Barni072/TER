\documentclass[11pt]{beamer}
\usetheme{Madrid}
\usefonttheme{serif}

\usepackage{tikz-cd}
\usepackage[utf8]{inputenc}
\usepackage[french]{babel}
\usepackage[T1]{fontenc}
\usepackage{amsmath}
\usepackage{amsfonts}
\usepackage{amssymb}
\usepackage{stmaryrd}
\usepackage{graphicx}
\usepackage[]{algorithm2e}

\setbeamertemplate{caption}[numbered]

\author{Barnabé Chabaux}
\title{Soutenance de TER}
\setbeamertemplate{navigation symbols}{}
\logo{\includegraphics[scale=0.08]{IF.jpg}}
\date{Implémentation en C/C++ de la résolution de systèmes linéaires à coefficients entiers}

\begin{document}
	\begin{frame}
		\titlepage
		\tableofcontents
	\end{frame}
	
	\section{Introduction}
	\begin{frame}{Introduction}
		Notations :
		\begin{itemize}
			\item $n$ : taille du système
			\item $c$ : taille (nombre de bits) des coefficients initiaux
			\item $L_i$ : $i$-ième ligne du 
			\item $C_j$ : $j$-ième colonne du système
			\item $a_{i,j}$ : coefficient de la $i$-ième ligne et de la $j$-ième colonne
			\item $b_i$ : coefficient de la $i$-ième ligne du second membre
		\end{itemize}
		Un système :
		$$\left \{
		\begin{array}{ccccccccccc}
			a_{1,1} x_1 &+ &\cdots &+ &a_{1,j} x_j &+ &\cdots &+ &a_{1,n} x_n &= & b_1\\
			\vdots & & & & \vdots & & & & \vdots & & \vdots\\
			a_{i,1} x_1 &+ &\cdots &+ &a_{i,j} x_j &+ &\cdots &+ &a_{i,n} x_n &= & b_i\\
			\vdots & & & & \vdots & & & & \vdots & & \vdots\\
			a_{n,1} x_1 &+ &\cdots &+ &a_{n,j} x_j &+ &\cdots &+ &a_{n,n} x_n &= & b_n
		\end{array}
		\right.
		$$
	\end{frame}
	
	\begin{frame}
		Un système peut aussi être représenté sous cette forme matricielle :
		\newline
		\begin{equation*}
			\begin{pmatrix}
				a_{1,1} & a_{1,2} & \cdots & a_{1,j} & \cdots & a_{1,n}&&b_1\\
				a_{2,1} & a_{2,2} & \cdots & a_{2,j} & \cdots & a_{2,n}&&b_2\\
				\vdots  & \vdots  & & \vdots & & \vdots&&\vdots\\
				a_{i,1} & a_{i,2} & \cdots & a_{i,j} & \cdots & a_{i,n}&&b_i\\
				\vdots  & \vdots  &  & \vdots & &\vdots&&\vdots\\
				a_{n,1} & a_{n,2} & \cdots & a_{n,j} & \cdots & a_{n,n}&&b_n
			\end{pmatrix}
		\end{equation*}
	\end{frame}
	
	\section{Pivot de Gauss}
	\begin{frame}{Pivot de Gauss}
		Pour $k$ allant de 1 à $n-1$ et $i$ allant de $k+1$ à $n$, on effectue l'opération suivante sur les lignes du système :
		\newline
		$$L_i \gets L_i - \frac{a_{i,k}}{a_{k,k}} L_k$$
		% Attention au cas où un pivot serait nul...
		\newline
		On obtient un système échelonné, c'est-à-dire un système dont la matrice est triangulaire supérieure.
		\vspace{0.5cm}
		\begin{equation*}
			\begin{pmatrix}
				17 & 2 & -3 & & 9\\
				4 & 7 & -8 & & -5\\
				1 & 0 & 5 & & 4
			\end{pmatrix}
			\longrightarrow
			\begin{pmatrix}
				17 & 2 & -3 & & 9\\
				0 & \frac{111}{17} & \frac{124}{17} & & \frac{-121}{17}\\
				0 & 0 & \frac{560}{111} & & \frac{371}{111}
			\end{pmatrix}
		\end{equation*}
		% Avec des ratiionnels, les coefficients deviennent trop grands et ce n'est pas gérable
	\end{frame}
	
	\section{Algorithme de Bareiss}
	\begin{frame}{Algorithme de Bareiss}
		Variante du pivot de Gauss, avec l'opération suivante sur les lignes du système (En posant $a_{0,0} = 1$) :
		$$L_i \gets \frac{a_{k,k} L_i - a_{i,k} L_k}{a_{k-1,k-1}}$$
		\vspace{0.8cm}
		\begin{equation*}
			\begin{pmatrix}
				17 & 2 & -3 & & 9\\
				4 & 7 & -8 & & -5\\
				1 & 0 & 5 & & 4
			\end{pmatrix}
			\longrightarrow
			\begin{pmatrix}
				17 & 2 & -3 & & 9\\
				0 & 111 & -124 & & -121\\
				0 & 0 & 560 & & 371
			\end{pmatrix}
		\end{equation*}
	\end{frame}
	
	\begin{frame}{Bareiss : coefficients entiers}
		À chaque utilisation de "l'opération", le déterminant du système est multiplié par $\frac{a_{k,k}}{a_{k-1,k-1}}$.
		\newline
		On a donc :
		$$\mbox{det}_{fin} = \frac{a_{1,1}^{n-1}}{a_{0,0}^{n-1}} \times \frac{a_{2,2}^{n-2}}{a_{1,1}^{n-2}} \times \frac{a_{3,3}^{n-3}}{a_{2,2}^{n-3}} \times \hdots \times \frac{a_{n-2,n-2}^2}{a_{n-3,n-3}^2} \times \frac{a_{n-1,n-1}}{a_{n-2,n-2}} \times \mbox{det}_{d\acute{e}but}$$
		$$\mbox{det}_{fin} = a_{1,1} \times a_{2,2} \times \hdots \times a_{n-1,n-1} \times \mbox{det}_{d\acute{e}but}$$
		D'autre part, comme la matrice en fin d'exécution est triangulaire supérieure :
		$$\mbox{det}_{fin} = a_{1,1} \times a_{2,2} \times \hdots \times a_{n-1,n-1} \times a_{n,n}$$
		Donc finalement : $\mbox{det}_{d\acute{e}but} = a_{n,n}$
		% Pour les autres coefficients, on applique ce raisonnement à une sous-matrice du système
		% Donc les coefficients sont des déterminants de sous-matrices (à coeffs entiers), donc sont entiers
	\end{frame}
	
	\begin{frame}{Bareiss : complexité}
		Borne de Hadamard :
		$$ \lvert \mbox{det}(A) \rvert \le \lVert C_1 \rVert_2 \times \hdots \times \lVert C_n \rVert_2$$
		\newline \newline
		\begin{itemize}
			\item Taille de $\lVert C_j \rVert_2 = \sqrt{a_{1,j}^2 + \hdots + a_{n,j}^2}$ : $O(c\mbox{ ln}(n))$
			\item Taille des coefficients : $O(cn\mbox{ ln}(n))$
			\item Coût des multiplications : $O(c^2n^2\mbox{ ln}(n)^2)$ (multiplication naïve) ou $O(cn\mbox{ ln}(n)^2)$ (FFT, à des $\mbox{ln}(\mbox{ln}(n))$ près)
			\item Coût total : $O(c^2 n^5 \mbox{ ln}(n)^2)$ (multiplications naïves) ou $O(c n^4 \mbox{ ln}(n)^2)$ (multiplications avec FFT)
		\end{itemize}
	\end{frame}
	
	\section{Méthode modulaire}
	\begin{frame}{Méthode modulaire}
		On va résoudre le système dans $\mathbb{Z}/p\mathbb{Z}$ pour diverses valeurs de $p$ (nombre premier de taille "raisonnable") (à l'aide du pivot de Gauss), et en déduire la solution rationnelle du système.
		\newline
		\newline
		Dans $\mathbb{Z}/p\mathbb{Z}$, les tailles des coefficients sont bornées, et le pivot de Gauss a donc un coût en $O(n^3)$.
		% Problèmes : Quand s'arrêter ? Comment "déduire" la solution ?
	\end{frame}
	
	\begin{frame}{Méthode modulaire : restes chinois}
		On cherche $x \in \mathbb{Z}/n_1n_2\mathbb{Z}$ tel que :
		\begin{equation*}
			\begin{cases}
				x \equiv x_1 \pmod {n_1}\\
				x \equiv x_2 \pmod {n_2}
			\end{cases}
		\end{equation*}
		Relation de Bézout minimale :
		\begin{equation*}
			\begin{cases}
				n_1 u + n_2 v = 1\\
				\lvert u \rvert \leq \frac{\lvert n_2 \rvert}{2}\\
				\lvert v \rvert  \leq \frac{\lvert n_1 \rvert}{2}
			\end{cases}
		\end{equation*}
		Méthode naïve : $x = x_1 v n_2 + x_2 u n_1$
		% Pas bon si $n_1$ est grand
		\newline
		Meilleure méthode : $x = x_1 + ((x_2 - x_1) u \pmod{n_2}) n_1$
	\end{frame}
	
	\begin{frame}{Méthode modulaire : reconstruction rationnelle}
		% Cette diapo est éclatée
		À partir d'un $b \in \mathbb{Z}/a\mathbb{Z}$, on cherche à obtenir un rationnel $\frac{num}{den}$, avec $num \in \mathbb{Z}$ et $den \in \mathbb{N}^*$ premiers entre eux, tel que :
		\begin{equation*}
			\begin{cases}
				num \  den^{-1} \equiv b \pmod a\\
				\lvert num \rvert < \frac{\sqrt{a}}{2}\\
				0 < den < \frac{\sqrt{a}}{2}\\
				den \wedge a = 1
			\end{cases}
		\end{equation*}
		Si une telle solution existe, on la trouve en utilisant une variante de l'algorithme d'Euclide étendu, où l'on s'arrête dès qu'on obtient un reste strictement inférieur à $\sqrt{a}$.
		% La solution est \frac{r_k}{v_k}
	\end{frame}
	
	\begin{frame}{Méthode modulaire : premier algorithme}
		\begin{itemize}
			\item Choisir un nombre premier $p$ de taille "raisonnable"
			\item Résoudre le système dans $\mathbb{Z}/p\mathbb{Z}$ (pivot de Gauss)
			\item Par les restes chinois et avec l'étape précédente, en déduire la solution du système dans $\mathbb{Z}/prod\mathbb{Z}$ (où $prod$ est le produit des nombres premiers utilisés depuis le début de l'algorithme)
			\item Effectuer une reconstruction rationnelle. Tant qu'on n'obtient pas le même résultat qu'à l'itération précédente, on recommence ces quatre étapes
		\end{itemize}
		% Il est techinquement nécessaire de "vérifier" la solution à la fin
		% La résolution rationnelle est trop coûteuse pour être effectuée à chaque étape, il faut trouver un autre moyen de s'arrêter
		% Complexité totale $O(c^3n^6 \mbox{ ln}(n)^3)$ (à cause des $O(cn ln(n))$ reconstructions rationnelles)
	\end{frame}
	
	\begin{frame}{Méthode modulaire : deuxième algorithme}
		Borne de Hadamard (avec second membre $b$) :
		$$ hada = \lVert C_1 \rVert_2 \times \hdots \times \lVert C_n \rVert_2 \times \lVert b \rVert_2$$
		Les coefficients étant obtenus par reconstruction rationnelle dans $\mathbb{Z}/prod\mathbb{Z}$, on a besoin d'avoir $\frac{\sqrt{prod}}{2} \geq hada $.
		% Expliquer que ceci vient des conditions sur le reconstruction rationnelle
		Dès que c'est le cas, il n'est plus nécessaire de faire des résolutions modulaires, et on peut effectuer une (seule) reconstruction rationnelle pour obtenir la solution du système.
		% Il n'est donc plus nécessaire de "vérifier" la solution obtenue
		\newline
		Il est nécessaire d'effectuer $O(c^2n^2 \mbox{ ln}(n)^2)$ itérations. 
		% La complexité (hors reconstruction rationelle) est donc en $O(cn^4 \mbox{ ln}(n)^2)$
		\newline
		La reconstruction rationnelle coûte $O(c^4n^5 \mbox{ ln}(n)^4)$.
		%$O(c^2n^2 \mbox{ ln}(n)^2)$ , $n$ fois
	\end{frame}
	
	\begin{frame}{Méthode modulaire : règle de Cramer}
		Notons $A_j$ la matrice du système (elle-même notée $A$) dont la $j$-ième colonne a été remplacée par le second membre $b$ :
		\begin{equation*}
			A_j = 
			\begin{pmatrix}
				a_{1,1} & a_{1,2} & \cdots & a_{1,j-1} & b_1 & a_{1,j+1} & \cdots & a_{1,n}\\
				a_{2,1} & a_{2,2} & \cdots & a_{2,j-1} & b_2 & a_{2,j+1} & \cdots & a_{2,n}\\
				\vdots  & \vdots  & & \vdots & \vdots & \vdots & & \vdots\\
				a_{i,1} & a_{i,2} & \cdots & a_{i,j-1} & b_i & a_{i,j+1} & \cdots & a_{i,n}\\
				\vdots  & \vdots  & & \vdots & \vdots & \vdots & & \vdots\\
				a_{n,1} & a_{n,2} & \cdots & a_{n,j-1} & b_n & a_{n,j+1} &\cdots & a_{n,n}
			\end{pmatrix}
			\in M_n(\mathbb{Z})
		\end{equation*}
		\newline
		On a alors : $ \forall j \in \llbracket1,n\rrbracket, x_j = \frac{\mbox{det}(A_j)}{\mbox{det}(A)}$
	\end{frame}
	
	\begin{frame}{Méthode modulaire : troisième algorithme}
		Tant que $\frac{prod}{2} < hada$ :
		% Le /2 vient de la représentation symétrique de Z/pZ
		\begin{itemize}
			\item Choisir un nombre premier $p$
			\item Échelonner/résoudre le système dans $\mathbb{Z}/p\mathbb{Z}$, et en déduire les déterminants de $A$ et des $A_j$ avec la formule de Cramer
			\item Par les restes chinois et avec l'étape précédente, en déduire les déterminants de $A$ et des $A_j$ dans $\mathbb{Z}/prod\mathbb{Z}$
		\end{itemize}
		Par la formule de Cramer, en déduire la solution du système
		% $prod$ est assez grand pour que le quotient \frac{det(A_j) mod prod}{det(A) mod prod} soit la solution du système
		% Cet algo ne fonctionne pas dans mon implémentation...
		% Complexité en $O(cn^4 \mbox{ ln}(n)^2)$
	\end{frame}
	
	\section{Mesures de temps de calcul, outils utilisés, et conclusion}
	\begin{frame}{Mesures de temps de calcul, outils utilisés, et conclusion}
		
		% Saut de ligne nécessaire (!?)
		{\footnotesize
		\begin{tabular}{|c||c|c|c|c|}
			\hline
			$\downarrow$ Taille ; Algorithme $\rightarrow$ & Gauss & Bareiss & Modulaire 1 & Modulaire 2\\
			\hline
			\hline
			$n$ = 5, $c$ = 96 & 0,000233 & 0,000093 & 0,00233 (35) & 0,000362 (42)\\
			\hline
			$n$ = 5, $c$ = 512 & 0,00144 & 0,000382 & 0,0957 (180) & 0,00293 (216)\\
			\hline
			$n$ = 5, $c$ = 2048 & 0,00736 & 0,00265 & 3,48 (719) & 0,0261 (860)\\
			\hline
			\hline
			$n$ = 50, $c$ = 12 & 0,0844 & 0,00934 & 0,0439 (45) & 0,00896 (47)\\
			\hline
			$n$ = 50, $c$ = 96 & 1,03 & 0,124 & 4,50 (337) & 0,103 (347)\\
			\hline
			$n$ = 50, $c$ = 512 & 12,4 & 1,72 & 493 (1796) & 1,35 (1829)\\
			\hline
			$n$ = 50, $c$ = 2048 & 102 & 13,4 &  $\times$& 18,3 (7321)\\
			\hline
			\hline
			$n$ = 200, $c$ = 12 & 26,1 & 1,36 & 5,28 (188) &  1,47 (199)\\
			\hline
			$n$ = 200, $c$ = 96 & 444 & 30,9 & 900 (1361) & 12,2 (1379)\\
			\hline
			$n$ = 200, $c$ = 512 & $\times$ & 388 & $\times$ & 117 (7729)\\
			\hline
			\hline
			$n$ = 700, $c$ = 12 & $\times$ & 298 & 644 (698) & 207 (734)\\
			\hline
		\end{tabular}
		}
		% Parler de la méthode en parallèle (qui n'est pas dans ce tableau), et de mes mesures douteuses
		\vspace{0.7cm}
		\newline
		Outils utilisés :
		\begin{itemize}
			\item Nombres entiers de la bibliothèque GNU MP
			\item Implémentation maison des nombres rationnels
			\item Représentation et génération aléatoire des systèmes linéaires
		\end{itemize}
		% Dire qu'il existe une méthode p-adique, non implémentée ici
	\end{frame}
	
\end{document}
